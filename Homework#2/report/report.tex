\documentclass{article}
\usepackage{amsmath}
\usepackage{amsfonts}
\usepackage{amssymb}
\usepackage{hyperref}
\usepackage{listings}
\usepackage[margin=2cm]{geometry}
\title{Homework\#2 Solutions}
\author{}
\date{}

\begin{document}

\maketitle

\section*{Problem 1: Iterative Methods (Gauss-Seidel Method)}

\subsection*{Given System of Equations}
\begin{align*}
10x_1 + 2x_2 - x_3 &= 27 \\
-3x_1 - 6x_2 + 2x_3 &= -61.5 \\
x_1 + x_2 + 5x_3 &= -21.5
\end{align*}

\subsection*{(a) Perform 1 iteration of the Gauss-Seidel method analytically}

Rearrange the equations to solve for each variable:
\begin{align*}
x_1 &= \frac{27 - 2x_2 + x_3}{10} \\
x_2 &= \frac{-61.5 + 3x_1 - 2x_3}{-6} \\
x_3 &= \frac{-21.5 - x_1 - x_2}{5}
\end{align*}

Assume initial guesses: \( x_1 = 0, x_2 = 0, x_3 = 0 \).

1st iteration:
\begin{align*}
x_1^{(1)} &= \frac{27 - 2(0) + 0}{10} = 2.7 \\
x_2^{(1)} &= \frac{-61.5 + 3(2.7) - 2(0)}{-6} = 8.9 \\
x_3^{(1)} &= \frac{-21.5 - 2.7 - 8.9}{5} = -6.62
\end{align*}

\section*{Problem 2: Newton-Raphson Method for Nonlinear Equations}

\subsection*{Given Equations}
\begin{align*}
y &= -x^2 + x + 0.75 \\
y + 1 &= x^2
\end{align*}

Initial guesses: \( x = 1.2, y = 1.2 \).

Rewrite as:
\begin{align*}
f_1(x, y) &= y + x^2 - x - 0.75 = 0 \\
f_2(x, y) &= x^2 - y - 1 = 0
\end{align*}

1. Evaluate Jacobian and functions:
   \[
   J = \begin{bmatrix}
   \frac{\partial f_1}{\partial x} & \frac{\partial f_1}{\partial y} \\
   \frac{\partial f_2}{\partial x} & \frac{\partial f_2}{\partial y}
   \end{bmatrix}
   =
   \begin{bmatrix}
   2x - 1 & 1 \\
   2x & -1
   \end{bmatrix}
   \]

2. Use Newton-Raphson update:
   \[
   \begin{bmatrix} x_{k+1} \\ y_{k+1} \end{bmatrix} = \begin{bmatrix} x_k \\ y_k \end{bmatrix} - J^{-1} \begin{bmatrix} f_1(x_k, y_k) \\ f_2(x_k, y_k) \end{bmatrix}
   \]

1st iteration:
\begin{align*}
J(1.2, 1.2) &= \begin{bmatrix} 1.4 & 1 \\ 2.4 & -1 \end{bmatrix} \\
f_1(1.2, 1.2) &= 1.2 + 1.44 - 1.2 - 0.75 = 0.69 \\
f_2(1.2, 1.2) &= 1.44 - 1.2 - 1 = -0.76
\end{align*}

Solve for \( \Delta x, \Delta y \):
\[
J^{-1} = \frac{1}{-1.4 - 2.4} \begin{bmatrix} -1 & -1 \\ -2.4 & 1.4 \end{bmatrix}
\]
\[
J^{-1} = \begin{bmatrix} 0.2631 & 0.2631 \\ 0.6315 & -0.3684 \end{bmatrix}
\]
\[
\Delta = J^{-1} \begin{bmatrix} 0.69 \\ -0.76 \end{bmatrix} = \begin{bmatrix} -0.018417 \\ 0.715719 \end{bmatrix}
\]

\[
\begin{bmatrix} x_{k+1} \\ y_{k+1} \end{bmatrix} = \begin{bmatrix} 1.2 \\ 1.2 \end{bmatrix} - \begin{bmatrix} -0.018417 \\ 0.715719 \end{bmatrix} = \begin{bmatrix} 1.218417 \\ 0.484281 \end{bmatrix}
\]

2nnd iteration:
   \begin{align*}
      J(1.218417, 0.484281) &= \begin{bmatrix} 1.436834 & 1 \\ 2.436834 & -1 \end{bmatrix} \\
      f_1(1.218417, 0.484281) &=  0.00040298588900022914\\
      f_2(1.218417, 0.484281) &= 0.0002599858890002249
   \end{align*}
Solve for \( \Delta x, \Delta y \):
\[
J^{-1} = \begin{bmatrix} 0.258153242 & 0.258153242 \\ 0.6290766271 & -0.3709233729 \end{bmatrix}
\]
% \[
% J^{-1} = \begin{bmatrix} 0.2631 & 0.2631 \\ 0.6315 & -0.3684 \end{bmatrix}
% \]
\[
\Delta = J^{-1} \begin{bmatrix} 0.00040298588900022914 \\ 0.0002599858890002249 \end{bmatrix} = \begin{bmatrix} 0.00017114831384532148789
   \\ 0.00015707416096679070974\end{bmatrix}
\]
\[
\begin{bmatrix} x_{k+1} \\ y_{k+1} \end{bmatrix} = \begin{bmatrix} 1.218417 \\ 0.484281 \end{bmatrix} - \begin{bmatrix} 0.00017114831384532148789 \\ 0.00015707416096679070974 \end{bmatrix} = \begin{bmatrix}  1.218399885 \\ 0.48271025839 \end{bmatrix}
\]

3rd iteration:
\begin{align*}
   J(1.218399885, 0.48271025839) &= \begin{bmatrix} 1.467997 & 1 \\ 2.4367997 & -1 \end{bmatrix} \\
   f_1(1.21839988, 0.48271025839) &= -0.0011913540259855804\\
   f_2(1.21839988, 0.48271025839) &= 0.0017880091940143394
   \end{align*}
   
   Solve for \( \Delta x, \Delta y \):
   \[
   J^{-1} =\begin{bmatrix} 0.258157826 & 0.258157826 \\ 0.629078913 & -0.370921087 \end{bmatrix}
   \]
   \[
   \Delta = J^{-1} \begin{bmatrix} -0.0011913540259855804 \\ 0.0017880091940143394 \end{bmatrix} = \begin{bmatrix} 0.00015403120104996912892 \\ -0.0014126660094749753355 \end{bmatrix}
   \]
   
   \[
   \begin{bmatrix} x_{k+1} \\ y_{k+1} \end{bmatrix} = \begin{bmatrix} 1.218399885 \\ 0.48271025839 \end{bmatrix} - \begin{bmatrix} 0.00015403120104996912892 \\ -0.0014126660094749753355  \end{bmatrix} = \begin{bmatrix} 1.2182458537989500308 \\ 0.4841229243994749754 \end{bmatrix}
   \]
   

\section*{Problem 3: LU Factorization and Matrix Inverse}

\subsection*{(a) Mass Balances for Reactors 2 and 3}
For reactor 2:
\[
-Q_{21} c_2 + Q_{12} c_1 - Q_{23} c_2 + Q_{32} c_3 - k V_2 c_2 = 0
\]
For reactor 3:
\[
-Q_{31} c_3 + Q_{13} c_1 - Q_{32} c_3 + Q_{23} c_2 - k V_3 c_3 = 0
\]

\subsection*{(b) Mass Balances for All Reactors}
Combine the equations for all reactors:
\[
\begin{aligned}
Q_{in} c_{in} - (Q_{12} + Q_{13} + k V_1) c_1 + Q_{21} c_2 + Q_{31} c_3 &= 0 \\
Q_{12} c_1 - (Q_{21} + Q_{23} + k V_2) c_2 + Q_{32} c_3 &= 0 \\
Q_{13} c_1 + Q_{23} c_2 - (Q_{31} + Q_{32} + k V_3) c_3 &= 0 \\
\end{aligned}
\]

\subsection*{(c) LU Factorization}
The system in matrix form is \( A\mathbf{c} = \mathbf{b} \).

\[
A = \begin{bmatrix}
-(Q_{12} + Q_{13} + k V_1) & Q_{21} & Q_{31} \\
Q_{12} & -(Q_{21} + Q_{23} + k V_2) & Q_{32} \\
Q_{13} & Q_{23} & -(Q_{31} + Q_{32} + k V_3) \\
\end{bmatrix}
\]
\[
\mathbf{b} = \begin{bmatrix}
-Q_{in} c_{in} \\
0 \\
0 \\
\end{bmatrix}
\]

Factorization is done in the python file script.py
\subsection*{(d) Answer Questions Using Matrix Inverse}
Values belowar are respective to reactor 1, 2, and 3.\\
(i) Steady-state concentrations:\\
73.62497731 62.00762389 67.4532583 \\

(ii) If the inflow in the second reactor is set to zero:\\
73.62497731 62.00762389 67.4532583 \\

(iii) If the inflow concentration to reactor 1 is doubled and reactor 2 is halved:\\
147.24995462 124.01524778 134.90651661

\section*{Problem 4: Gauss Elimination}

Given system:
\[ -3x_2 + 7x_3 = 4 \]
\[ x_2 + 2x_2 - x_3 = 0 \]
\[ 5x_1 - 2x_2 = 3 \]

\subsection*{(a) Compute the determinant analytically.}
The matrix would be: 

\[ \begin{pmatrix} 0 & -3 & 7 \\ 1 & 2 & -1 \\ 5 & -2 & 0 \end{pmatrix} \]

\[
   \det(A) = 0 \cdot \begin{vmatrix} 2 & -1 \\ -2 & 0 \end{vmatrix} - (-3) \cdot \begin{vmatrix} 1 & -1 \\ 5 & 0 \end{vmatrix} + 7 \cdot \begin{vmatrix} 1 & 2 \\ 5 & -2 \end{vmatrix} = -69
\]
\subsection*{(b) Solve using Cramer's rule.}
we replace the first column with to get the matrix \( A_1 \) for \( x_1 \):
\[\begin{pmatrix}
   4 & -3 & 7 \\ 0 & 2 & -1 \\ 3 & -2 & 0
\end{pmatrix}\]
\[
   x_1 = \frac{\det(A_1)}{\det(A)} = 0.59420
\]
we replace the second column with to get the matrix \( A_2 \) for \( x_2 \):
\[\begin{pmatrix}
   0 & 4 & 7 \\ 1 & 0 & -1 \\ 5 & 3 & 0
\end{pmatrix}\]
\[
   x_2 = \frac{\det(A_2)}{\det(A)} = -0.14493
\]
we replace the third column with to get the matrix \( A_3 \) for \( x_3 \):
\[\begin{pmatrix}
   0 & -3 & 4 \\ 1 & 2 & 0 \\ 5 & -2 & 3
\end{pmatrix}\]
\[
   x_3 = \frac{\det(A_3)}{\det(A)} = 0.56521
\]
\subsection{Gaussian Elimination using Partial Pivoting:}
\[
   \begin{pmatrix}
      0 & -3 & 7 & \|4 \\
      1 & 2 & -1 & \|0 \\
      5 & -2 & 0 & \|3
   \end{pmatrix}
\]

We look at the first column and realize 5 is the largest and do row-swaps to make it the pivot element, which results in the matrix:
switch the rows 1 and 3 to make the pivot element 5:
\[
   \begin{pmatrix}
      5 & -2 & 0 & \|3 \\
      1 & 2 & -1 & \|0 \\
      0 & -3 & 7 & \|4
   \end{pmatrix}
\]


now, we need the transformation \[r_2 = r_2 - \frac{1}{5}r_1\]
\[
   \begin{pmatrix}
      5 & -2 & 0 & \|3 \\
      0 & -2.4 & -1 & \|-0.6 \\
      0 & -3 & 7 & \|4
   \end{pmatrix}
\]

Next, we need the transformation \[r_3 = r_3 - \frac{1}{3}r_2\times2.4\]
\[
   \begin{pmatrix}
      5 & -2 & 0 & \|3 \\
      0 & -3 & 7 & \|4 \\
      0 & 0 & 4.6 & \|2.6
   \end{pmatrix}
\]

Now, we can back-substitute to get the values of \(x_1, x_2, x_3\):
\[x_3 = \frac{2.6}{4.6} = 0.5652178913\]
\[x_2 = \frac{4 - 7 \times 0.5652178913}{-3} = -0.1449275362\]
\[x_1 = \frac{3 + 2 \times 0.1449275362}{5} = 0.5942028986\]

Finding the determinant:
A=\[
   \begin{pmatrix}
      5 & -2 & 0 & \\
      0 & -3 & 7 & \\
      0 & 0 & 4.6 &
   \end{pmatrix}
\]

\[
   \det(A) = 5 \times -3 \times 4.6 = -69
\]
which is  equal to the derminant found in part (a).

(d) Substitute your results back into the original equations to check your solution:
\[
   -3 \times -0.1449275362 + 7 \times 0.5652178913 \approx 4
\]

\section*{Problem 5: Eigenvalues}

Given matrix:
\[ \begin{pmatrix} 20 & 3 & 2 \\ 3 & 9 & 4 \\ 2 & 4 & 12 \end{pmatrix} \]

\subsection*{(a) Determine eigenvalues from characteristic polynomial}

we can get the characteristic polynomial of a 3$\times$3 matrix by expanding the below equation: 

\[\lambda^3-s_1\lambda^2+s_2\lambda-\det A\]

Where S1 and S2 are the sum of the diagonal elements and sum of principal minors. 

Here, \[s_1=41\]
\[s_2 =108-16 + 240 -4+180-9 = 499\]
\[\det A = 1744\]

The equationw would then become: 
\[\lambda^3-41\lambda^2+499\lambda- 1744 = 0\]
Obtained Eingenvalues are 21.72915618 13.18232626  6.08851756
\subsection*{(b) Use the power method to find the largest eigenvalue and compare this with the result from (a)}

The power method is used to find the largest eigenvalue of a matrix. Perform 3 iterations analytically:

1. **Choose an initial vector \( x_0 \):**
\[ x_0 = \begin{pmatrix} 1 \\ 1 \\ 1 \end{pmatrix} \]

2. **Iteration 1:**
\[ y_1 = Ax_0 = \begin{pmatrix} 20 & 3 & 2 \\ 3 & 9 & 4 \\ 2 & 4 & 12 \end{pmatrix} \begin{pmatrix} 1 \\ 1 \\ 1 \end{pmatrix} = \begin{pmatrix} 25 \\ 16 \\ 18 \end{pmatrix} \]

\[ \text{Normalize } y_1: \]
\[ 25 \times \begin{pmatrix} 1 \\ 0.64 \\ 0.73 \end{pmatrix} \]

3. **Iteration 2:**
\[ y_2 = Ax_1 = \begin{pmatrix} 20 & 3 & 2 \\ 3 & 9 & 4 \\ 2 & 4 & 12 \end{pmatrix} \begin{pmatrix} 1 \\ 0.64 \\ 0.73 \end{pmatrix} = \begin{pmatrix} 23.36 \\ 11.64 \\ 13.2 \end{pmatrix} \]

\[ \text{Normalize } y_2: \]
\[ 23.36 \times \begin{pmatrix} 1 \\ 0.4982876712 \\ 0.5650684932 \end{pmatrix} \]

4. **Iteration 3:**
\[ y_3 = Ax_2 = \begin{pmatrix} 20 & 3 & 2 \\ 3 & 9 & 4 \\ 2 & 4 & 12 \end{pmatrix} \begin{pmatrix} 1 \\ 0.4982876712 \\ 0.5650684932 \end{pmatrix} = \begin{pmatrix} 22.625 \\ 9.744863014 \\ 10.7738726 \end{pmatrix} \]

The largest eigenvalue found using the power method is approximately 22.625, which is close to the exact eigenvalue of 22.251 found using the characteristic polynomial.

\subsection*{(c) Use the inverse power method to find the smallest eigenvalue and compare this with the result from (a)}

The python script gave the smallest eigen value as 6.088517562501856, which is close to the exact eigenvalue of 6.08852 found using the characteristic polynomial.


\end{document}
