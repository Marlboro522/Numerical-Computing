\documentclass[12pt]{article}
\usepackage{amsmath, amssymb, graphicx}

\title{CS 4600/5600 Numerical Computing\\ Homework 1}
\author{Student Name}
\date{Fall 2024}

\begin{document}

\maketitle

\section*{Problem 2: Roots - Open Methods}

\subsection*{(b) Newton-Raphson Method}

Given the function:
\[
f(x) = x^3 - 6x^2 + 11x - 6.1
\]

We need to determine the largest positive root using the Newton-Raphson method, starting with \(x_0 = 3.5\).

The Newton-Raphson iteration formula is:
\[
x_{n+1} = x_n - \frac{f(x_n)}{f'(x_n)}
\]

First, we find the derivative of \(f(x)\):
\[
f'(x) = 3x^2 - 12x + 11
\]

Starting with \(x_0 = 3.5\):
\[
f(3.5) = (3.5)^3 - 6(3.5)^2 + 11(3.5) - 6.1 = 0.775
\]
\[
f'(3.5) = 3(3.5)^2 - 12(3.5) + 11 = 1.75
\]

First iteration:
\[
x_1 = 3.5 - \frac{0.775}{1.75} \approx 3.0571
\]

Second iteration:
\[
f(3.0571) = (3.0571)^3 - 6(3.0571)^2 + 11(3.0571) - 6.1 \approx 0.1123
\]
\[
f'(3.0571) = 3(3.0571)^2 - 12(3.0571) + 11 \approx 1.1714
\]
\[
x_2 = 3.0571 - \frac{0.1123}{1.1714} \approx 2.9612
\]

Third iteration:
\[
f(2.9612) = (2.9612)^3 - 6(2.9612)^2 + 11(2.9612) - 6.1 \approx 0.0053
\]
\[
f'(2.9612) = 3(2.9612)^2 - 12(2.9612) + 11 \approx 1.0783
\]
\[
x_3 = 2.9612 - \frac{0.0053}{1.0783} \approx 2.9563
\]

The largest positive root after three iterations is approximately \(x \approx 2.9563\).

\subsection*{(c) Modified Secant Method}

Using the same function \(f(x) = x^3 - 6x^2 + 11x - 6.1\), we apply the modified secant method with \(x_0 = 3.5\) and \(\delta = 0.01\).

The modified secant method iteration formula is:
\[
x_{n+1} = x_n - \frac{\delta x_n f(x_n)}{f(x_n + \delta x_n) - f(x_n)}
\]

First iteration:
\[
f(3.5) = 0.775
\]
\[
f(3.5 + 0.01 \cdot 3.5) = f(3.535) = 0.771925
\]
\[
x_1 = 3.5 - \frac{0.01 \cdot 3.5 \cdot 0.775}{0.771925 - 0.775} \approx 3.0357
\]

Second iteration:
\[
f(3.0357) = 0.1087
\]
\[
f(3.0357 + 0.01 \cdot 3.0357) = f(3.0661) = 0.107092
\]
\[
x_2 = 3.0357 - \frac{0.01 \cdot 3.0357 \cdot 0.1087}{0.107092 - 0.1087} \approx 2.9634
\]

Third iteration:
\[
f(2.9634) = 0.0074
\]
\[
f(2.9634 + 0.01 \cdot 2.9634) = f(2.9930) = 0.0068
\]
\[
x_3 = 2.9634 - \frac{0.01 \cdot 2.9634 \cdot 0.0074}{0.0068 - 0.0074} \approx 2.9567
\]

Fourth iteration:
\[
f(2.9567) = 0.0027
\]
\[
f(2.9567 + 0.01 \cdot 2.9567) = f(2.9862) = 0.0026
\]
\[
x_4 = 2.9567 - \frac{0.01 \cdot 2.9567 \cdot 0.0027}{0.0026 - 0.0027} \approx 2.9559
\]

Fifth iteration:
\[
f(2.9559) = 0.0009
\]
\[
f(2.9559 + 0.01 \cdot 2.9559) = f(2.9855) = 0.0008
\]
\[
x_5 = 2.9559 - \frac{0.01 \cdot 2.9559 \cdot 0.0009}{0.0008 - 0.0009} \approx 2.9556
\]

The largest positive root after five iterations is approximately \(x \approx 2.9556\).

\section*{Problem 4: Optimization}

Given the function:
\[
f(x) = 4x - 1.8x^2 + 1.2x^3 - 0.3x^4
\]

\subsection*{(b) Prove the function is concave for all values of \(x\)}

A function is concave if its second derivative is negative for all values of \(x\).

First, we find the first derivative of \(f(x)\):
\[
f'(x) = 4 - 3.6x + 3.6x^2 - 1.2x^3
\]

Now, we find the second derivative:
\[
f''(x) = -3.6 + 7.2x - 3.6x^2
\]

For \(f''(x)\) to be negative for all values of \(x\), we need to check its sign:
\[
f''(x) = -3.6 + 7.2x - 3.6x^2
\]

Let's analyze the roots of \(f''(x)\):
\[
-3.6 + 7.2x - 3.6x^2 = 0
\]
\[
3.6x^2 - 7.2x + 3.6 = 0
\]
\[
x^2 - 2x + 1 = 0
\]
\[
(x - 1)^2 = 0
\]
\[
x = 1
\]

At \(x = 1\):
\[
f''(1) = -3.6 + 7.2(1) - 3.6(1)^2 = -3.6 + 7.2 - 3.6 = 0
\]

The second derivative is zero at \(x = 1\), so we need to analyze the behavior around \(x = 1\):
For \(x < 1\), \(f''(x)\) is positive.
For \(x > 1\), \(f''(x)\) is negative.

Since \(f''(x)\) changes sign, the function is not concave for all values of \(x\).

\subsection*{(c) Differentiate the function and use a root-location method to solve for the maximum \(f(x)\) and the corresponding value of \(x\)}

First, we find the critical points by setting the first derivative to zero:
\[
f'(x) = 4 - 3.6x + 3.6x^2 - 1.2x^3 = 0
\]

We can use the Newton-Raphson method to find the root:
Starting with \(x_0 = 1\):
\[
f'(1) = 4 - 3.6(1) + 3.6(1)^2 - 1.2(1)^3 = 0
\]
\[
f''(1) = -3.6 + 7.2(1) - 3.6(1)^2 = 0
\]

Since \(f''(1) = 0\), we use the modified Newton-Raphson method:
\[
x_{n+1} = x_n - \frac{f'(x_n)}{f''(x_n)}
\]

Second iteration:
\[
f''(0.5) = -3.6 + 7.2(0.5) - 3.6(0.5)^2 = -3.6 + 3.6 - 0.9 = -0.9
\]
\[
x_1 = 0.5 - \frac{f'(0.5)}{f''(0.5)} \approx 0.5
\]

The maximum \(f(x)\) is at \(x = 1\).

\subsection*{Problem 5: Optimization}

Given the function:
\[
f(x_1, x_2) = \frac{1}{2}(x_1^2 - x_2)^2 + \frac{1}{2}(1 - x_1)^2
\]

\subsection*{(a) Minimum point}

To find the minimum point, we need to find the gradient and set it to zero:
\[
\nabla f = \left( \frac{\partial f}{\partial x_1}, \frac{\partial f}{\partial x_2} \right) = 0
\]

First, we find the partial derivatives:
\[
\frac{\partial f}{\partial x_1} = 2x_1(x_1^2 - x_2) - (1 - x_1)
\]
\[
\frac{\partial f}{\partial x_2} = -(x_1^2 - x_2)
\]

Setting the gradient to zero:
\[
2x_1(x_1^2 - x_2) - (1 - x_1) = 0
\]
\[
-(x_1^2 - x_2) = 0
\]

Solving the system of equations, we get:
\[
x_1^2 = x_2
\]
\[
2x_1(x_1^2 - x_2) - (1 - x_1) = 0
\]

Simplifying, we find the minimum point is:
\[
(x_1, x_2) = (1, 1)
\]

\subsection*{(b) Newton's Method Iteration}

Using the starting point \(\begin{pmatrix} 2 \\ 2 \end{pmatrix}\), we perform one iteration of Newton's method for minimizing \(f\).

The Hessian matrix is:
\[
H = \begin{pmatrix}
6x_1^2 - 2x_2 + 1 & -2x_1 \\
-2x_1 & 1
\end{pmatrix}
\]

The gradient at \(\begin{pmatrix} 2 \\ 2 \end{pmatrix}\) is:
\[
\nabla f = \begin{pmatrix}
2x_1(x_1^2 - x_2) - (1 - x_1) \\
-(x_1^2 - x_2)
\end{pmatrix}
= \begin{pmatrix}
2(2)(4 - 2) - (1 - 2) \\
-(4 - 2)
\end{pmatrix}
= \begin{pmatrix}
6 \\
-2
\end{pmatrix}
\]

The Hessian at \(\begin{pmatrix} 2 \\ 2 \end{pmatrix}\) is:
\[
H = \begin{pmatrix}
6(2)^2 - 2(2) + 1 & -2(2) \\
-2(2) & 1
\end{pmatrix}
= \begin{pmatrix}
17 & -4 \\
-4 & 1
\end{pmatrix}
\]

The Newton step is:
\[
\Delta x = -H^{-1} \nabla f
\]

We need to invert the Hessian:
\[
H^{-1} = \frac{1}{17 \cdot 1 - (-4)^2} \begin{pmatrix}
1 & 4 \\
4 & 17
\end{pmatrix}
= \frac{1}{1} \begin{pmatrix}
1 & 4 \\
4 & 17
\end{pmatrix}
= \begin{pmatrix}
1 & 4 \\
4 & 17
\end{pmatrix}
\]

The Newton step is:
\[
\Delta x = - \begin{pmatrix}
1 & 4 \\
4 & 17
\end{pmatrix} \begin{pmatrix}
6 \\
-2
\end{pmatrix}
= - \begin{pmatrix}
1 \cdot 6 + 4 \cdot (-2) \\
4 \cdot 6 + 17 \cdot (-2)
\end{pmatrix}
= - \begin{pmatrix}
6 - 8 \\
24 - 34
\end{pmatrix}
= - \begin{pmatrix}
-2 \\
-10
\end{pmatrix}
= \begin{pmatrix}
2 \\
10
\end{pmatrix}
\]

The new point after one iteration is:
\[
\begin{pmatrix}
2 \\
2
\end{pmatrix}
+ \begin{pmatrix}
2 \\
10
\end{pmatrix}
= \begin{pmatrix}
4 \\
12
\end{pmatrix}
\]

\end{document}